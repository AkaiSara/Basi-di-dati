\subsubsection{Per Ogni}

Dato lo schema relazionale $R(A,B,C)$, trovare gli $A$ per i quali tutti i $C$ sono positivi.

\begin{displaymath}
  \pi_{A}(R) - \pi_{A}(\sigma_{C \leq 0}(R))
\end{displaymath}

A tutti gli $A$ si vogliono togliere quegli $A$ per cui $C$ è negativo, ottenendo così
gli $A$ per i quali tutti i $C$ sono positivi. \\

\noindent
Dato lo schema relazionale $R(A,B,C)$, trovare gli $A$ per i quali tutti i $C$ sono uguali.

\begin{displaymath}
  \pi_{A}(R) - \pi_{A}(R \bowtie_{A=A^1 \land C \neq C^1}(\rho_{A^1,B^1,C^1 \Leftarrow A,B,C}(R)))
\end{displaymath}

Il theta join permette di trovare tutti gli elementi di $A$ che non hanno tutti i $C$ uguali.
Sottraendo ciò che troviamo dal join con l'insieme di tutti gli elementi, otteniamo gli $A$ per
i quali tutti i $C$ sono uguali.
