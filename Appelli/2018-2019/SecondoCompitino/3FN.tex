\documentclass[12pt,a4paper]{article}

\usepackage[english,italian]{babel}
\usepackage[utf8]{inputenc}
\usepackage[T1]{fontenc}

%------Math------
\usepackage{mathtools} %math package
\usepackage{amssymb} %pacchetto per i simboli di matematica
\usepackage{amsmath} %pacchetto di matematica
\usepackage{amsthm} %pacchetto di matematica\raggedbottom

%-------Margini------
\usepackage{geometry}
\geometry{a4paper,top=1cm,bottom=2cm,left=1.5cm,right=1.5cm,heightrounded}
\raggedbottom


\begin{document}
    \author{Andrea Longo \and Sara Righetto}
    \title{Terza Forma Normale}
    \maketitle

    Dato lo schema di relazione R(A,B,C,D,E,F) con dipendenze: \newline

    \noindent
    BD $\rightarrow$ F \newline
    FE $\rightarrow$ BA \newline
    F $\rightarrow$ C \newline
    B $\rightarrow$ E \newline
    C $\rightarrow$ A \newline
    E $\rightarrow$ CD \newline

    \noindent
    a. Trovare la copertura ridotta. \newline
    [Sostituire le dipendenze con altre equivalenti con un singolo attributo al secondo membro.] \newline

    \noindent
    BD $\rightarrow$ F \newline
    FE $\rightarrow$ B \newline
    FE $\rightarrow$ A \newline
    F $\rightarrow$ C \newline
    B $\rightarrow$ E \newline
    C $\rightarrow$ A \newline
    E $\rightarrow$ C \newline
    E $\rightarrow$ D \newline

    \noindent
    [Verificare l'esistenza di attributi eliminabili al primo membro.] \newline

    \noindent 
    BD $\rightarrow$ F diventa B $\rightarrow$ E e E $\rightarrow$ D $\Rightarrow$ B $\rightarrow$ F \newline

    \noindent
    [Eliminazione degli attributi ridondanti.] \newline

    \noindent %B-F
    B $\rightarrow$ F calcolo la chiusura di B (senza considerare B $\rightarrow$ F). \newline
    B$^+ =$ \{ E,C,D,A \} \newline

    \noindent %FE -B
    FE $\rightarrow$ B calcolo la chiusura di FE (senza considerare FE $\rightarrow$ B). \newline
    FE$^+ =$ \{ F,E,A,C,D \} \newline

    \noindent %FE-A
    FE $\rightarrow$ A calcolo la chiusura di FE (senza considerare FE $\rightarrow$ A). \newline
    A $\in$ FE$^+ =$ \{ F,E,B,C,D,A \} la dipendenza è implicata dalle altre e può essere eliminata. \newline 

    \noindent %F-C
    F $\rightarrow$ C calcolo la chiusura di F (senza considerare F $\rightarrow$ C). \newline
    F$^+ =$ \{ F \} \newline

    \noindent %B-E
    B $\rightarrow$ E calcolo la chiusura di B (senza considerare B $\rightarrow$ E). \newline
    B$^+ =$ \{ B,F,C,A \} \newline

    \noindent %C-A
    C $\rightarrow$ A calcolo la chiusura di C (senza considerare C $\rightarrow$ A). \newline
    C$^+ =$ \{ C \} \newline

    \noindent %E-C
    E $\rightarrow$ C calcolo la chiusura di E (senza considerare E $\rightarrow$ C). \newline
    E$^+ =$ \{ E,D \} \newline

    \noindent %E-D
    E $\rightarrow$ D calcolo la chiusura di E (senza considerare e $\rightarrow$ D). \newline
    E$^+ =$ \{ E,C,A \} \newline

    \noindent
    La copertura ridotta (dopo aver rimosso FE $\rightarrow$ A) \newline

    \noindent
    B $\rightarrow$ F \newline
    FE $\rightarrow$ B \newline
    F $\rightarrow$ C \newline
    B $\rightarrow$ E \newline
    C $\rightarrow$ A \newline
    E $\rightarrow$ C \newline
    E $\rightarrow$ D \newline

    \noindent
    b. Trovare le chiavi di R partendo dalla copertura ridotta. \newline
    [Raccolgo tutte le implicazioni per lettera] \\

    \noindent
    B$^+ =$ \{ B,F,C,A,D,E \} include tutti gli attributi quindi è \textbf{chiave}\newline
    FE$^+ =$ \{ B,F,C,A,D,E \} include tutti gli attributi quindi è \textbf{chiave}\newline
    F$^+ =$ \{ F,C,A \} \emph{non} è chiave \newline
    C$^+ =$ \{ C,A \} \emph{non} è chiave \newline
    E$^+ =$ \{ E,D,C,A \} \emph{non} è chiave \newline

    \noindent
    c. Dire quali dipendenze violano 3FN. \newline
    [Per essere valide devono avere una chiave al primo membro oppure almeno una parte di chiave al secondo membro] \newline

    \noindent
    B $\rightarrow$ F non viola 3FN \newline
    FE $\rightarrow$ B non viola 3FN  \newline
    F $\rightarrow$ C viola 3FN \newline
    B $\rightarrow$ E non viola 3FN \newline
    C $\rightarrow$ A viola 3FN \newline
    E $\rightarrow$ C viola 3FN \newline
    E $\rightarrow$ D viola 3FN \newline

    \noindent
    d. Normalizzare lo schema in 3FN \newline
    [Calcolo partizionamento, cioè membri con la stessa chiusura.] \newline

    \( 
    \left.
    \begin{array}{l}
    B \rightarrow F \\
    FE \rightarrow B\\
    B \rightarrow E \\
    \end{array}
    \right] R_1 $ dato che $ $B$^+ $/$ $FE$^+ = $\{ B,F,C,A,D,E \}$ 
    \)
    \newline

    \(
    \left.
    \begin{array}{l}
    F \rightarrow C \\
    \end{array}
    \right] R_2 $ dato che $ $F$^+ = $\{ F,C,A \}$
    \) 
    \newline
    
    \(
    \left.
    \begin{array}{l}
    C \rightarrow A\\
    \end{array}
    \right] R_3 $ dato che C$^+ =$\{ C,A \}$
    \)
    \newline

    \(
    \left.
    \begin{array}{l}
    E \rightarrow C \\
    E \rightarrow D \\
    \end{array}
    \right] R_4$ dato che E$^+ = $\{ E,D,C,A\}$ 
    \)
    \newline

    \noindent
    [Per ogni R$_N$ segno gli attributi coinvolti e come chiave metto il primo membro.] \newline

    \noindent
    R$_1$ (B,E,F) con chiave B, FE. \newline
    R$_2$ (E,C,D) con chiave E. \newline
    R$_3$ (C,A) con chiave C. \newline
    R$_4$ (F,C) con chiave F. \newline



\end{document}

