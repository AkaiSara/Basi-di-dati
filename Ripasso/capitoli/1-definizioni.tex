\section{Definizioni}
\subsection{Modello Relazionale}
\subsubsection{Definizione chiave e superchiave}

Un sottoinsieme $K$ di attributi è superchiave per uno schema di relazione $r$
se, per ogni coppia di tuple distinte, i valori assunti dalle
tuple in corrispondenza non sono tutti uguali.

\noindent
Una chiave è una superchiave minimale, ovvero una superchiave la quale,
tolto un qualunque attributo, non è più superchiave. In altre parole, non
esiste un'altra superchiave $K^1$ di $r$ che sia contenuta in $K$ come sottoinsieme proprio.

\noindent
Ogni chiave è superchiave,
ma in generale non vale il viceversa.

\subsection{Algebra e Calcolo Relazionale}
\subsubsection{Definizione di Join}
Operatore che permette di correlare dati contenuti in relazioni diverse,
confrontando i valori contenuti in esse. Esiste in più varianti.

\subsubsection{Definizione di Natural Join}
Operatore binario che correla dati in relazioni diverse, sulla base di valori
uguali, in attibuti con lo stesso nome.

\noindent
Simbolo: $\bowtie$

\noindent
Proprietà:

\begin{enumerate}
  \item Commutatività: $r1 \bowtie r2 = r2 \bowtie r1$
  \item Associatività $r1 \bowtie (r2 \bowtie r3) = (r1 \bowtie r2) \bowtie r3$
  \item Se gli insiemi $X_1$ e $X_2$ di attributi di due tuple sono uguali, allora i
  Natural Join coincide con un'intersezione.
  \item Se gli insiemi $X_1$ e $X_2$ di attributi di due tuple sono disgiunti, allora i
  Natural Join coincide con il prodotto cartesiano.
\end{enumerate}

\subsubsection{Definizone di Theta Join}
Operatore definito come il prodotto cartesiano seguito da una selezione, nel modo seguente
(dove $F$ è una formula proposizionale utilizzabile in una selezione, e dove le relazioni
$r_1$ e $r_2$ non hanno attributi in comune):

\begin{displaymath}
  r_1 \bowtie_{F} r_2 = \sigma_{F}(r_1 \bowtie r_2)
\end{displaymath}

\subsubsection{Definizione di Equi Join}
L'Equi Join non è altro che un Theta Join in cui la condizione di selezione $F$
sia una congiunzione di uguaglianza, con un attributo della prima relazione $r1$
e uno della seconda $r2$.

\subsection{Progettazione Concettuale}
\subsubsection{Definizione strategia di progetto top-down}

Nella strategia top-down, lo schema concettuale viene prodotto mediante raffinamenti
successivi a partire da uno schema iniziale che, pur descrivendo tutte le specifiche,
resta astratto. Tale schema viene a via a via raffinato aumentando il livello di dettagli,
ma mantiene le medesime informazioni. Tutti gli aspetti presenti nello schema finale
sono presenti a ogni livello di raffinamento.

\noindent
PRO: il progettista può inizialmente descrivere tutte le specifiche dei dati
trascurandone i dettagli

\noindent
CONTRO: è necessario possedere sin dall'inizio una visione globale di tutte le componenti
del sistema

\subsubsection{Definizione strategia di progetto bottom-up}

Nella strategia bottom-up si suddividono le specifiche in modo da sviluppare
diversi schemi elementari ma dettagliati, che successivamente vengono integrati
tra di loro. Tale strategia favorisce lo sviluppo in team.
