\section{Definizioni}
\subsection{Modello Relazionale}
\subsubsection{Definizione chiave e superchiave}

Un sottoinsieme $K$ di attributi è superchiave per uno schema di relazione $r$
se, per ogni coppia di tuple distinte, i valori assunti dalle
tuple in corrispondenza non sono tutti uguali.

\noindent
Una chiave è una superchiave minimale, ovvero una superchiave la quale,
tolto un qualunque attributo, non è più superchiave. In altre parole, non
esiste un'altra superchiave $K^1$ di $r$ che sia contenuta in $K$ come sottoinsieme proprio.

\noindent
Ogni chiave è superchiave,
ma in generale non vale il viceversa.

\subsubsection{Definizione Vincolo di integrità}

Si tratta di una proprietà che deve essere soddisfatta dalle istanze che rappresentano
informazioni corrette per l'applicazione. Ogni vincolo può essere visto come un predicato
che associa ad ogni istanza il valore $vero$ o $falso$. Se e solo se il predicato assume
il valore $vero$, allora si dice che l'istanza soddisfa il vincolo.
Lo scopo di questi vincoli è di vietare una situazione indesiderata all'interno della
base di dati.

\subsubsection{Definizione Vincolo intrarelazionale}
Vincolo il cui soddisfacimento è definito rispetto a singole relazioni della base di dati.
Talvolta, il vincolo riguarda le tuple separatemente le une dalle altre.

\subsubsection{Definizione Vincolo interrelazionale}
Vincolo che coinvolge più relazioni. Permette di vietare una situazione indesiderata ad esempio
richiedendo che un numero di matricola compaia nella relazione ESAMI solo se compare anche
nella relazione STUDENTI.

\subsubsection{Definizione Vincolo di integrità referenziale}
Un vincolo di integrità referenziale tra un insieme di attributi $X$ di due relazioni
$R_{1}$ e $R_{2}$ è soddisfatto se i valori su $X$ di ciascuna tupla dell'istanza di $R_{1}$
compaiono come valori della chiave primaria dell'istanza di $R_{2}$.

\subsection{Algebra e Calcolo Relazionale}

\subsubsection{Definizione di Selezione}
Operatore monadico il cui risultato ha gli stessi attributi dell'operando, e contiene solo
le ennuple dell'operando che soddisfano la condizione specificata. La condizione deve essere
un'espressione booleana. \\
Si dice che la selezione effettua una decomposizione orizzontale. \\
Simbolo: $\sigma_{F}(R)$

\subsubsection{Definizione di Proiezione}
Operatore monadico il cui risultato ha solo la parte specificata $Y$ degli attributi $X$ dell'operando,
ma contiene tutte le ennuple dell'operando considerate solo sui valori di $Y$. \\
Si dice che la proiezione effettua una decomposizione verticale. \\
Simbolo: $\Pi_{Y_{1},...,Y^{N}}(R)$

\subsubsection{Cardinalità Proiezione}
Definedo $N$ la cardinalità della proiezione:

\begin{enumerate}
  \item se non esistono duplicati nelle ennuple dlel'operando $r_{1}$ allora $ N = \left\vert{r_{1}}\right\vert$
  \item altrimenti, a causa dell'eliminazione dei duplicati si ottiene $ 0 \leq N < \left\vert{r_{1}}\right\vert $
\end{enumerate}

\subsubsection{Definizione di Join}
Operatore che permette di correlare dati contenuti in relazioni diverse,
confrontando i valori contenuti in esse. Esiste in più varianti.

\subsubsection{Definizione di Natural Join}
Operatore binario che correla dati in relazioni diverse, sulla base di valori
uguali, in attibuti con lo stesso nome.

\noindent
Simbolo: $\bowtie$

\noindent
Proprietà:

\begin{enumerate}
  \item Commutatività: $r1 \bowtie r2 = r2 \bowtie r1$
  \item Associatività $r1 \bowtie (r2 \bowtie r3) = (r1 \bowtie r2) \bowtie r3$
  \item Se gli insiemi $X_1$ e $X_2$ di attributi di due tuple sono uguali, allora i
  Natural Join coincide con un'intersezione.
  \item Se gli insiemi $X_1$ e $X_2$ di attributi di due tuple sono disgiunti, allora i
  Natural Join coincide con il prodotto cartesiano.
\end{enumerate}

\subsubsection{Definizone di Theta Join}
Operatore definito come il prodotto cartesiano seguito da una selezione, nel modo seguente
(dove $F$ è una formula proposizionale utilizzabile in una selezione, e dove le relazioni
$r_1$ e $r_2$ non hanno attributi in comune):

\begin{displaymath}
  r_1 \bowtie_{F} r_2 = \sigma_{F}(r_1 \bowtie r_2)
\end{displaymath}

\subsubsection{Definizione di Equi Join}
L'Equi Join non è altro che un Theta Join in cui la condizione di selezione $F$
sia una congiunzione di uguaglianza, con un attributo della prima relazione $r1$
e uno della seconda $r2$.

\subsection{Cardinalità Join}
Definedo $N$ la cardinalità del join:

\begin{enumerate}
  \item se il join di $r_{1}$ e $r_{2}$ è completo allora $ 0 \leq N \leq max(\left\vert{r_{1}}\right\vert, \left\vert{r_{2}}\right\vert)$
  \item se il join coinvolge una chiave di $r_{2}$, allora $ 0 \leq N \leq \left\vert{r_{1}}\right\vert$
  \item se il join coinvolge una chiave di $r_{2}$ ed $\exists$ un vincolo di integrità referenziale
        tra un attributo di $r_{1}$ e la chiave di $r_{2}$, allora $ N = \left\vert{r_{1}}\right\vert $
\end{enumerate}

\subsection{Progettazione Concettuale}
\subsubsection{Definizione strategia di progetto top-down}

Nella strategia top-down, lo schema concettuale viene prodotto mediante raffinamenti
successivi a partire da uno schema iniziale che, pur descrivendo tutte le specifiche,
resta astratto. Tale schema viene a via a via raffinato aumentando il livello di dettagli,
ma mantiene le medesime informazioni. Tutti gli aspetti presenti nello schema finale
sono presenti a ogni livello di raffinamento.

\noindent
PRO: il progettista può inizialmente descrivere tutte le specifiche dei dati
trascurandone i dettagli

\noindent
CONTRO: è necessario possedere sin dall'inizio una visione globale di tutte le componenti
del sistema

\subsubsection{Definizione strategia di progetto bottom-up}

Nella strategia bottom-up si suddividono le specifiche in modo da sviluppare
diversi schemi elementari ma dettagliati, che successivamente vengono integrati
tra di loro. Tale strategia favorisce lo sviluppo in team.

\subsubsection{Definizione strategia di progetto inside-out}

Si tratta di un caso particolare della bottom-up. Inizialmente si individuano solo alcuni
concetti importanti e poi si procede, a partire da questi, a "macchi d'olio"; si rappresentano
cioè prima i concetti in relazione con quelli iniziali, per poi muoversi verso quelli più
lontani, in una sorta di "navigazione" tra le specifiche.