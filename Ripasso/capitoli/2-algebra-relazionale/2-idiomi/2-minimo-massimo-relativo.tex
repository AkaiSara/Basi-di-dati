\subsubsection{Minimo e Massimo relativo}

Dato lo schema relazionale $R(A,B)$, trovare per ogni $A$ il minimo/massimo in R.
Si supponga di voler determinare il massimo $B$ in $A$:

\begin{displaymath}
  \pi_{A,B}(R) - \pi_{A,B}(R \bowtie_{A=A^1 \land B<B^1}(\rho{A^1,B^1 \leftarrow A,B}(R)))
\end{displaymath}

È molto simile al massimo assoluto. Il theta join in questo caso seleziona tutti i valori
minimi di $B$ per ogni attributo $A$.

\noindent
Esempio con la relazione \textbf{ESAMI} in cui $A$ sia Studente e $B$ sia Voto:

\begin{gather*}
  S1 := \rho_{Studente^1,Voto^1 \leftarrow Studente,Voto}(\textbf{ESAMI}) \\
  \pi_{Nome,Voto}(\textbf{ESAMI}) - \pi_{Nome,Voto}(\textbf{ESAMI} \bowtie_{Studente=Studente^1 \land Voto<Voto^1}(S1))
\end{gather*}